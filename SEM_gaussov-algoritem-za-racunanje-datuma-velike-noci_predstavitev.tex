\documentclass[a4paper,12pt]{article}

\usepackage[slovene]{babel}
\usepackage[utf8]{inputenc}
\usepackage[T1]{fontenc}
\usepackage{lmodern}

\usepackage{amsfonts,amssymb,amsmath}
\usepackage{graphicx}
\usepackage{makecell}


\def\qed{$\hfill\Box$}   % konec dokaza
\def\qedm{\qquad\Box}   % konec dokaza v matematičnem načinu
\newtheorem{izrek}{Izrek}
\newtheorem{trditev}{Trditev}
\newtheorem{posledica}{Posledica}
\newtheorem{lema}{Lema}
\newtheorem{pripomba}{Pripomba}
\newtheorem{definicija}{Definicija}
\newtheorem{zgled}{Zgled}

\setlength{\tabcolsep}{8pt}

\title{Gaussov algoritem za računanje datuma velike noči \\ 
\Large Seminar}
\author{Marjetka Zupan \\
Fakulteta za matematiko in fiziko \\
Oddelek za matematiko}
\date{\today}

\begin{document}

\section*{Uvod}
Anekdota pravi, da si mati Johanna Carla Friedricha Gaussa ni zapomnila 
točnega datuma rojstva svojega sina. Vedela je le, da je bil rojen v sredo, 
8 dni pred praznikom Jezusovega vnebohoda leta 1777. Da bi ugotovil, kdaj 
je bil rojen, je Gauss razvil algoritem za izračun datuma velike noči. 

Namen njegovega dela ni razprava že znanih običajnih postopkov določanja 
tega datuma. Ti so po Gaussovih besedah enostavni, če bralec pozna pomen 
nekaterih pojmov, ki se pri formulaciji postopka navadno uporabljajo (zlato 
število, epakta, velikonočni mejnik, Sončev cikel, nedeljska črka), in ima 
pred seboj potrebne pomožne tabele. Želel je poiskati način za določanje 
tega datuma brez dodatnih pripomočkov (tabel), zgolj z uporabo enostavnih 
računskih operacij, ki si jih bo vsak, ki se mu to zdi vredno truda, z 
lahkoto zapomnil.

Svoj algoritem je pri 23 letih objavil v članku ``Berechnung des Osterfestes 
(Izračun datuma velike noči).''


\section*{Kdaj praznujemo veliko noč?}

Da bomo lahko razumeli Gaussov algoritem za določanje datuma velike noči, 
moramo najprej vedeti, kdaj sploh praznujemo veliko noč. Naivna splošno znana 
„definicija“ velike noči je „prva nedelja po prvi spomladanski polni luni,“ a 
ta ni povsem natančna in med drugim tudi ne enolična.
Velika noč se je do leta 325 praznovala na dan, ki ga je sproti določil papež, 
leta 325 pa so na podlagi opazovanj določili postopek za izračun datuma prve 
spomladanske polne lune in veliko noč praznovali na prvo nedeljo po njej. Ker 
ta postopek ni bil povsem natančen, se napovedani datumi ne ujemajo nujno z 
dejanskimi datumi, na katere lahko vidimo polno luno. Ker so pri določanju 
datuma velike noči pomembni le datumi iz napovedi, namesto o prvi spomladanski 
polni luni raje govorimo o pashalni polni luni.

Za lažje razumevanje uvedimo še pojem luninega meseca. To je čas med dvema 
zaporednima praznima lunama.

Leta 325 je bil v uporabi julijanski koledar. Tega sestavljajo običajna leta, 
dolga 365 dni, in prestopna leta, dolga 366 dni, pri čemer je vsako s 4 
deljivo leto prestopno. V povprečju tako eno leto v julijanskem koledarju 
traja 365,25 dni. Postopek določanja pashalne polne lune temelji na metonskem 
ciklu, opazki, da je 19 let v julijanskem koledarju dolgih natanko toliko kot 
235 luninih mesecev. Tako se datumi praznih lun in posledično tudi datumi 
pashalnih polnih lun vsakih 19 let ponavljajo. Vsakemu letu torej lahko glede 
na položaj v tem ciklu pripišemo število med 1 in 19. To število imenujemo 
zlato število in ga izračunamo po formuli $n_a = (A \bmod 19) + 1$, pri čemer 
smo z $A$ označili leto, za katero želimo izračunati zlato število. Datum 
pashalne polne lune potem lahko preberemo iz tabele, veliko noč pa praznujemo 
na prvo nedeljo po tem datumu.

\begin{center}
    \begin{tabular}{| c c | c c | c c |}
        \hline
        \makecell{zlato \\ število} & \makecell{pashalna \\ polna luna} & \makecell{zlato \\ število} & \makecell{pashalna \\ polna luna} & \makecell{zlato \\ število} & \makecell{pashalna \\ polna luna} \\ \hline
        1 & 5.\ april & 8 & 18.\ april & 15 & 1.\ april \\  
        2 & 25.\ marec & 9 & 7.\ april & 16 & 21.\ marec \\
        3 & 13.\ april & 10 & 27.\ marec & 17 & 9.\ april \\
        4 & 2.\ april & 11 & 15.\ april & 18 & 29.\ marec \\
        5 & 22.\ marec & 12 & 4.\ april & 19 & 17.\ april \\
        6 & 10.\ april & 13 & 24.\ marec & & \\
        7 & 30.\ marec & 14 & 12.\ april & & \\ \hline
    \end{tabular}
\end{center}

Določanje datuma velike noči se zaplete leta 1582, ko so zaradi neskladja med 
dolžino Zemljinega obhoda okoli Sonca in dolžine julijanskega leta uvedli nov, 
gregorijanski koledar. Tudi tega sestavljajo običajna, 365 dni dolga leta in 
prestopna, 366 dni dolga leta, le da prestopno leto nastopi vsako s 4 deljivo 
leto, ki ni hkrati deljivo tudi s 100, pri tem pa so leta, ki so deljiva s 400, 
vseeno prestopna. Nov koledar naj bi to neskladje zmanjšal.
Poleg reforme koledarja pa so leta 1582 uvedli tudi reformo določanja datuma 
velike noči. Metonski cikel se je namreč izkazal za netočnega, poleg tega pa 
je bil z gregorijanskim koledarjem še slabše usklajen kot z julijanskim.

V ta namen so v postopek določanja datuma velike noči uvedli epakto. To je 
mera za starost lune na določen dan in je enaka številu dni od dneva, ko je 
luna prvič vidna (to je praviloma dan po prazni luni). V julijanskem koledarju 
epakto leta definiramo kot epakto lune na 22. marec, v gregorijanskem 
koledarju pa kot epakto lune ob začetku leta.

Epakto lahko izračunamo s pomočjo zlatega števila.
Če označimo $e' = (11 \cdot (a - 1)) \bmod 30$, je ob predpostavki, da za 
julijanski koledar velja metonski cikel, epakta enaka $e'$ razen v primeru, ko 
je $e' = 0$ in namesto 0 raje vzamemo 30 (včasih ljudje res niso marali ničle).
Ker imamo le 19 možnih zlatih števil, je tudi različnih vrednosti epakte 
julijanskega leta 19 -- 1, 3, 4, 6, 7, 9, 11, 12, 14, 15, 17, 18, 20, 22, 23, 25, 26, 28 in 30.

V gregorijanskem koledarju računanje epakte prilagodimo in uvedemo nekaj 
popravkov, da bo izračunana vrednost čim bližje pravi, astronomsko določeni. 
Epakto v gregorijanskem koledarju določimo po naslednjem postopku:

\begin{enumerate}
    \item Uporabimo formulo za julijanski koledar 
        $e_j = (11 \cdot (a - 1)) \bmod 30$
    \item Uvedemo solarno enačbo $S = \lfloor \frac{3 \cdot C}{4} \rfloor $

        Ta odraža razliko med julijanskim in gregorijanskim koledarjem in se 
        vsako leto, ki je prestopno v julijanskem, ne pa tudi v gregorijanskem 
        koledarju, poveča za 1.
    \item Uvedemo lunarno enačbo $L = \lfloor \frac{8 \cdot C + 5}{25} \rfloor $
        
        Ta odraža napako pri uporabi metonskega cikla v gregorijanskem 
        koledarju in se vsakih 2500 let osemkrat poveča za 1.
    \item Izračunamo $e_g' = e_j - S + L + 8$. 
    \item Podamo gregorijansko epakto $e_g' \bmod 30$, pri čemer ponovno 
        namesto 0 vzamemo 30.
\end{enumerate}

Kljub temu da je možnih julijanskih epakt le 19, gregorijanska epakta lahko 
zavzame poljubno vrednost med 1 in 30.

Datum velike noči po gregorijanskem koledarju določimo glede na epakto. Datum 
pashalne polne lune preberemo iz tabele, veliko noč pa praznujemo na nedeljo, 
ki temu datumu sledi.

\begin{center}
    \begin{tabular}{| c c | c c | c c |}
        \hline
        epakta & \makecell{pashalna \\ polna luna} & epakta & \makecell{pashalna \\ polna luna} & epakta & \makecell{pashalna \\ polna luna} \\ \hline
        1 & 12.\ april & 11 & 2.\ april & 21 & 23.\ marec \\  
        2 & 11.\ april & 12 & 1.\ april & 22 & 22.\ marec \\
        3 & 10.\ april & 13 & 31.\ marec & 23 & 21.\ marec \\
        4 & 9.\ april & 14 & 30.\ marec & 24 & 18.\ april \\
        5 & 8.\ april & 15 & 29.\ marec & 25 & 17.\ ali 18.\ april \\
        6 & 7.\ april & 16 & 28.\ marec & 26 & 17.\ april \\
        7 & 6.\ april & 17 & 27.\ marec & 27 & 16.\ april \\
        8 & 5.\ april & 18 & 26.\ marec & 28 & 15.\ april \\
        9 & 4.\ april & 19 & 25.\ marec & 29 & 14.\ april \\
        10 & 3.\ april & 20 & 24.\ marec & 30 & 13.\ april \\ \hline
    \end{tabular}
\end{center} %TODO sklici na tabele

Kot vidimo v tabeli, sta pri epakti 25 možna dva različna datuma. 
Pravega izberemo na enega od ekvivalentnih načinov:
\begin{itemize}
    \item Če trenutno stoletje vsebuje leto z epakto 24, izberemo 17. april, 
        sicer pa 18.
    \item Če je zlato število strogo večje od 11, izberemo 17. april, sicer 
        pa 18.
\end{itemize}

Dokaz, da sta oba načina ekvivalentna, opustimo, lahko pa ga preberete v članku.
\begin{align*}
    \exists \text{ leto z epakto } e_g = 24 &\Leftrightarrow \exists \text{ leto z } d_l = 29 \\
    &\Leftrightarrow d_l = (19 a_l + M) \bmod 30 = 29 \\
    &\Leftrightarrow 19 a_l \bmod 30 + M = 29 \\
    &\Leftrightarrow M = 29 - 19 a_l \bmod 30
\end{align*}

Dano leto $A$ ima epakto $e_g = 25$, torej $d = 28$ in ena od:
\begin{itemize}
    \item $19 a \bmod 30 = 29 \land M = 29$
    \item $M = 28 - (19 a \bmod 30)$
\end{itemize}

Če prvo: $19 a \bmod 30 = 19 (a - 11 + 11) \bmod 30 = 29 + 19 (a - 11) \bmod 30 = 29$.
Sledi: $19 (a - 11) \bmod 30 = 0$ in ob upoštevanju $0 \leq a \leq 18$ $a=11$.

Če drugo: $M$ je za obe leti enak, ker sta obe v istem stoletju, torej
\begin{align*}
    29 - 19 a_l \bmod 30 = 28 - 19 a \bmod 30 \\
    19 (a_l - a) \bmod 30 = 1 \\
    19 (a_l - a + 11 - 11) \bmod 30 = 1 \\
    19 (a_l - (a - 11)) \bmod 30 = 0 \\
    a_l - (a - 11) \bmod 30 = 0
\end{align*}

Ob upoštevanju $0 \leq a, a_l \leq 18$ dobimo $a_l = a - 11$ in $a \geq 11$ oz.\ $a > 10$.

Obrat: Če je za leto $A$ $a > 10$ in $e_g = 25$, je za to leto $d = 28$ in imata leti 
$A - 11$ in $A + 8$ epakto $e_g = 24$.
Eno izmed njiju je zagotovo v istem stoletju kot $A$.
Res, velja: $a_l = (A - 11) \bmod 19 = (A + 8) \bmod 19 = a - 11$ (upoštevali smo $a > 10$)
\begin{align*}
    e_g &= (23 - d_l) \bmod 30 \\
        &= (23 - (19 a_l + M) \bmod 30) \bmod 30 \\
        &= (23 - 19 (a - 11) - M) \bmod 30 \\
        &= (22 - 19 a - M) \bmod 30 \\
        &= (22 - (19 a + M)) \bmod 30 = (22 - 28) \bmod 30 = 24
\end{align*}

\textsc{Primer:} velika noč letos

\section*{Gaussov algoritem}

Gauss v svojem delu podaja enostavnejši algoritem za izračun datuma velike 
noči tako po julijanskem kot tudi po gregorijanskem koledarju. Najprej vpelje 
nekatere oznake: 
\begin{align*} %TODO leto pišem leto ali dam oznako
    a &= A \bmod 19 \\
    b &= A \bmod 4 \\
    c &= A \bmod 7 \\
    d &= (19 \cdot a + M) \bmod 30 \\
    e &= (2 \cdot b + 4 \cdot c + 6 \cdot d + N) \bmod 7,
\end{align*}
pri čemer sta $M$ in $N$ za računanje datuma v julijanskem koledarju konstanti, 
za računanje datuma v gregorijanskem koledarju pa ju za leta med $100 k$ in 
$100 k + 99$ izračunamo na naslednji način: 
\begin{align*}
    p &= \lfloor \frac{8k + 13}{25} \rfloor \\
    q &= \lfloor \frac{k}{4} \rfloor \\
    M &= (15 + k - p - q) \bmod 30 \\
    N &= (4 + k - q) \bmod 7,
\end{align*}
za leta med 1582 in 1599 pa vzamemo $k = 16$.

Pri tem imamo v gregorijanskem koledarju ti dve izjemi:
\begin{itemize}
    \item Če za datum velike noči dobimo 26. april, namesto tega vedno vzamemo 
        19. april.
    \item Če dobimo $d = 28$ in $e = 6$ ter število $11M+11$ pri deljenju s 
        30 da ostanek, manjši od 19, namesto 25. aprila vzamemo 18.
\end{itemize}

\textsc{Primer:} velika noč letos

\section*{Dokaz}

Razmislimo, da podan algoritem velja. Začnimo pri julijanskem koledarju. Prvi 
korak algoritma je deljenje po modulu 19, ki ga najdemo tudi v prej opisanem 
postopku. Z njim namreč določimo zlato število leta.

Datum pashalne polne lune smo potem prebrali s tabele.

To tabelo zapišimo malo drugače, in sicer namesto datuma pashalne polne lune 
zapišimo število dni od 21. marca do nje.

\begin{center}
    \begin{tabular}{| c c | c c | c c |}
        \hline
        \makecell{zlato \\ število} & \makecell{pashalna \\ polna luna} & \makecell{zlato \\ število} & \makecell{pashalna \\ polna luna} & \makecell{zlato \\ število} & \makecell{pashalna \\ polna luna} \\ \hline
        1 & 15 & 8 & 28 & 15 & 11 \\  
        2 & 4 & 9 & 17 & 16 & 0 \\
        3 & 23 & 10 & 6 & 17 & 19 \\
        4 & 12 & 11 & 25 & 18 & 8 \\
        5 & 1 & 12 & 14 & 19 & 27 \\
        6 & 20 & 13 & 3 & & \\
        7 & 9 & 14 & 22 & & \\ \hline
    \end{tabular}
\end{center}

Opazimo, da število v naslednji vrstici dobimo z alternirajočim prištevanjem 
19 in odštevanjem 11, od koder sklepamo na prištevanje števila 19 in deljenje 
po modulu 30. Ničlo najdemo pri $n_a = 16$, od koder dobimo formulo:
\begin{align*}
    d &= 19 \cdot (n_a - 16) \bmod 30 \\
    &= 19 \cdot (a - 15) \bmod 30 \\
    &= (19 \cdot a - 19 \cdot 15) \bmod 30 \\
    &= (19 \cdot a + 15) \bmod 30
\end{align*}
Datum pashalne polne lune v julijanskem koledarju je torej 21 + d. marec, pri 
čemer je za 21 + d > 31 to seveda d - 10. april. Podobno bomo predpostavili 
tudi v nadaljevanju in ne bomo ločevali primerov.

Določili smo datum pashalne polne lune, vendar veliko noč praznujemo prvo 
nedeljo po tem datumu, tako da še nismo končali. Dobljenemu datumu moramo 
prišteti še eno od števil 1, 2, 3, 4, 5, 6, 7. Veliko noč torej praznujemo 
$21 + d + e'$. marca, $e' = 1, 2, 3, 4, 5, 6, 7$, oz. $22 + d + e$. marca, 
$e = 0, 1, 2, 3, 4, 5, 6$, kjer $e$ oz. $e'$ izberemo tako, da je ta dan 
nedelja. Opazimo, da temu pogoju ustreza parameter $e$ v Gaussovem algoritmu.

Utemeljimo, da ta res vrne takšno število, da je $22 + d + e$. marec nedelja. 
Za začetek potrebujemo datum, za katerega vemo, da je bila nedelja. Ker 
določamo datum velike noči po julijanskem koledarju, vzamemo datum pred 
4. oktobrom 1582, npr. 19. april 1500 (da je to res bila nedelja, mi boste 
morali verjeti, lahko pa sami preverite v koledarju). Med 19. aprilom 1500 
in $22 + d + e$. marcem leta $A$ je $d + e - 28 + i + 365(A - 1500)$ dni, 
kjer $i$ šteje prestopna leta med tema dvema datumoma (29. februar je v vsakem 
primeru pred datumom v algoritmu). 

Velja: $i = \lfloor \displaystyle \frac{(A - 1500)}{4} \rfloor = \frac{A - 1500 - (A \bmod 4)}{4}$

Število dni med 19. aprilom 1500 in $22 + d + e$. marcem leta $A$ je torej enako:
\begin{align*}
    &d + e - 28 + \frac{(A - 1500 - b)}{4} + 365(A - 1500) &/ + (28 + \frac{7}{4} (A - 1500 - b)) \\
    &d + e + 2 A - 3000 - 2b + 365(A - 1500) &/ + (550494 - 364 A) \\
    &d + e + 3 A - 6 - 2b &/ - (3 A - 3c) \\
    &d + e + 3c - 6 - 2b &/ - (7d + 7c) \\
    &e - 2b - 4c - 6d - 6
\end{align*}

Ker želimo, da je $22 + d + e$. marec leta $A$ nedelja, mora biti 
$(e - 2b - 4c - 6d - 6) \bmod 7 = 0$, to pa velja natanko tedaj, ko je
$e = (2b + 4c + 6d + 6) \bmod 7$, kar je ravno Gaussov $e$ za $N=6$.

Pokažimo, da algoritem velja tudi za gregorijanski koledar.

Ponovno začnemo z določanjem zlatega števila. Kot prej opazimo:
\begin{align*}
    n_a &= a + 1 \\
    e_j &= (11 \cdot a) \bmod 30 \\
    e_g &= (((11 \cdot a) \bmod 30) - S + L + 8) \bmod 30 \\
        &= (11a - S + L + 8) \bmod 30 \\
        &= (11a - \lfloor \frac{3 (k + 1)}{4} \rfloor + \lfloor \frac{8 (k + 1) + 5}{25} \rfloor + 8) \bmod 30 \\
        &= (11a - k + \lfloor \frac{k}{4} \rfloor + \lfloor \frac{8 (k + 1) + 5}{25} \rfloor + 8) \bmod 30
\end{align*}

Tabelo, iz katere glede na epakto preberemo datum pashalne polne lune, ponovno
zapišemo drugače, in sicer kot prej namesto datuma pashalne polne lune 
pišemo število dni od 21. marca do nje.

\begin{center}
    \begin{tabular}{| c c | c c | c c |}
        \hline
        epakta & \makecell{pashalna \\ polna luna} & epakta & \makecell{pashalna \\ polna luna} & epakta & \makecell{pashalna \\ polna luna} \\ \hline
        1 & 22 & 11 & 12 & 21 & 2 \\  
        2 & 21 & 12 & 11 & 22 & 1 \\
        3 & 20 & 13 & 10 & 23 & 0 \\
        4 & 19 & 14 & 9 & 24 & 28 \\
        5 & 18 & 15 & 8 & 25 & 27 ali 28 \\
        6 & 17 & 16 & 7 & 26 & 27 \\
        7 & 16 & 17 & 6 & 27 & 26 \\
        8 & 15 & 18 & 5 & 28 & 25 \\
        9 & 14 & 19 & 4 & 29 & 24 \\
        10 & 13 & 20 & 3 & 30 & 23 \\ \hline
    \end{tabular}
\end{center} %TODO sklici na tabele

Opazimo, da je datum pashalne polne lune (razen v primeru 
$e_g = 25$ ali $e_g = 24$) $21 + ((23 - e_g) \bmod 30)$. marec.

$21 + ((23 - ((11a - k + \lfloor \frac{k}{4} \rfloor + \lfloor \frac{8 (k + 1) + 5}{25} \rfloor + 8) \bmod 30)) \bmod 30)$

$21 + ((23 - 11a + k - \lfloor \frac{k}{4} \rfloor - \lfloor \frac{8 (k + 1) + 5}{25} \rfloor - 8) \bmod 30)$

$21 + ((15 - 11a + k - p - q) \bmod 30)$

$21 + ((11a + M) \bmod 30)$

$21 + d$

Kot prej tudi tu utemeljimo, da je izbira $e$ prava. Tokrat moramo za 
izhodišče izbrati neki datum po 15. oktobru 1582. Kot Gauss bomo tudi mi 
vzeli nedeljo, 21. marec 1700.
Dobimo:
\begin{align*}
    &d + e + 1 + \textstyle \frac{A - 1700 - (A \bmod 4)}{4} - C + 18 + \frac{C - 17 - ((C - 1) \bmod 4)}{4} + 365 (A - 1700) \\
    &d + e + 20 + \textstyle \frac{A - 1700 - b}{4} - k + \frac{k - 16 - (k \bmod 4)}{4} + 365(A - 1700) \\
    &d + e + 20 + \textstyle \frac{A - b}{4} - 425 - k + \lfloor \frac{k}{4} \rfloor - 4 + 365A - 2065 &/ \frac{7}{4} (A - b) \\
    &d + e + 4 + 3A - 2b - k + q &/ -(3A - 3c) \\
    &e - (2 b + 4 c + 6 d + N) \\
    &\Leftrightarrow e = (2b + 4c + 6d + 6) \bmod 7
\end{align*}

Ustavimo se še pri izjemah, ki ju navaja Gaussov algoritem za izračun datuma 
velike noči po gregorijanskem koledarju. Spomnimo se, da smo pri izpeljavi 
formule v gregorijanskem koledarju predpostavili, da $e_g$ ni enak 24 ali 25. 
Ta dva primera pokrivata ravno ti dve izjemi.

Prva izjema pravi: ``Če za datum velike noči dobimo 26.\ april, namesto tega vedno 
vzamemo 19.\ april.''

Gaussov algoritem vrne 26. april natanko tedaj, ko je $d + e - 9 = 26$. Ker je 
$d$ ostanek pri deljenju s 30, $e$ pa ostanek pri deljenju s 7, je edina možnost 
za to $d = (19 a + M) \bmod 30 = 29$ in $e = 6$. 
Potem velja:
\begin{align*}
    e_g &= (11a - k + p + q + 8) \bmod 30 = \\
        &= (-19 a + 23 - (15 + k - p - q)) \bmod 30 = \\
        &= (23 - (19 a + M)) \bmod 30 = \\
        &= (23 - ((19 a + M) \bmod 30)) \bmod 30 = \\
        &= (23 - d) \bmod 30 = (23 - 29) \bmod 30 = 24
\end{align*}
V tem primeru je torej epakta 24. Ker je $e_g = (23 - d) \bmod 30$, velja ekvivalenca.

Za $d = 29$ oz.\ $e_g = 24$ je datum pashalne polne lune po tabeli 18. april, 
Gaussov algoritem pa za datum pashalne polne lune določi 19. april. Ko prištejemo 
še $e$, Gaussov algoritem vrne prvo nedeljo po tem datumu, zato je razlika le 
pri $e=6$. V tem primeru je datum velike noči 19.\ april.

Druga izjema pravi: ``Če dobimo $d = 28$ in $e = 6$ ter število $11M + 11$ pri deljenju
s 30 da ostanek, manjši od 19, namesto 25. aprila vzamemo 18.''

Razmislimo, kdaj dobimo $d=28$. Velja:
\begin{align*}
    d &= (19 a + M) \bmod 30 = \\
    &= ((19 a \bmod 30) + (M \bmod 30)) \bmod 30, 
\end{align*}
torej imamo dve možnosti:
\begin{itemize}
    \item $19 a \bmod 30 = M \bmod 30 = 29$
    \item $(19 a \bmod 30) + (M \bmod 30) = 28$
\end{itemize}

V prvem primeru je $(11M + 11) \bmod 30 = (11 \cdot 29 + 11) \bmod 30 = 0$ in 
$19 a \bmod 30 = 29$, torej $a = 11$ in $n_a = 12$. 
 
($19 a \bmod 30 = 19 (a - 11 + 11) \bmod 30 = 29 + (19 (a - 11) \bmod 30)$)

V drugem primeru je 
\begin{align*}
    M \bmod 30 &= 28 - (19 a \bmod 30) \\
    (11 M + 11) \bmod 30 &= (11 (28 - (19 a \bmod 30)) + 11) \bmod 30 = \\
        &= (19 - (11 \cdot 19 a \bmod 30)) \bmod 30 = \\
        &= (19 - 11 \cdot 19 a) \bmod 30 = \\
        &= (19 - 11 \cdot 19 \cdot (a - 11 + 11)) \bmod 30 = \\
        &= (-11 \cdot 19 \cdot (a - 11)) \bmod 30 = \\
        &= (a - 11) \bmod 30
\end{align*}

Opazimo, da je $(11M + 11) \bmod 30 = (a - 11) \bmod 30 < 19$ natanko tedaj, ko je 
$a > 10$ oz.\ $n_a = a + 1 > 11$.

\begin{align*}
    e_g &= (11a - k + p + q + 8) \bmod 30 = \\
        &= (-19 a + 23 - (15 + k - p - q)) \bmod 30 = \\
        &= (23 - (19 a + M)) \bmod 30 = \\
        &= (23 - ((19 a + M) \bmod 30)) \bmod 30 = \\
        &= (23 - d) \bmod 30 = (23 - 28) \bmod 30 = 25
\end{align*}

Iz $e_g = 25$ sledi $d = 28$. Če je poleg tega zlato števlio strogo večje od 11, je 
tudi $(11M + 11) \bmod 30 < 19$ in imamo pashalno polno luno 18.\ aprila (v obeh 
primerih). Sicer 17. Ponovno je razlika v datumu velike noči le za $e = 6$.


\textsc{Primer:} Gaussov rojstni dan




\end{document}
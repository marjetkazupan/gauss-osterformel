\documentclass[a4paper,12pt]{article}

\usepackage[slovene]{babel}
\usepackage[utf8]{inputenc}
\usepackage[T1]{fontenc}
\usepackage{lmodern}

\usepackage{amsfonts,amssymb,amsmath}
\usepackage{graphicx}
\usepackage{makecell}
\usepackage{url}


\def\qed{$\hfill\Box$}   % konec dokaza
\def\qedm{\qquad\Box}   % konec dokaza v matematičnem načinu
\newtheorem{izrek}{Izrek}
\newtheorem{trditev}{Trditev}
\newtheorem{posledica}{Posledica}
\newtheorem{lema}{Lema}
\newtheorem{pripomba}{Pripomba}
\newtheorem{definicija}{Definicija}
\newtheorem{zgled}{Zgled}

\title{Gaussov algoritem za računanje datuma velike noči \\ 
\Large Seminar}
\author{Marjetka Zupan \\
Fakulteta za matematiko in fiziko \\
Oddelek za matematiko}
\date{\today}

\begin{document}


%%%%%%%%%%%%%%%%%%%%%%%%%%%%%%%%%%%%%%%%%%%%%%%%%%%%%%%%%%%%%%%%%%%%%


\maketitle
\begin{abstract}
   %TODO
\end{abstract}

%TODO ključne besede


%%%%%%%%%%%%%%%%%%%%%%%%%%%%%%%%%%%%%%%%%%%%%%%%%%%%%%%%%%%%%%%%%%%%%

\section{Uvod}

Anekdota pravi, da si mati Johanna Carla Friedricha Gaussa ni zapomnila 
točnega datuma rojstva svojega sina. Vedela je le, da je bil rojen v sredo, 
8 dni pred praznikom Jezusovega vnebohoda leta 1777. Da bi ugotovil, kdaj 
je bil rojen, je Gauss razvil algoritem za izračun datuma velike noči. 

Sam v uvodu pravi, da namen njegovega dela ni razprava že znanih običajnih postopkov določanja
tega datuma. Ti so po Gaussovih besedah enostavni, če bralec pozna pomen 
nekaterih pojmov, ki se pri formulaciji postopka navadno uporabljajo 
(zlato število, epakta, velikonočni mejnik, Sončev cikel, nedeljska črka), 
in ima pred seboj potrebne pomožne tabele.
Želel je poiskati način za določanje tega datuma brez dodatnih pripomočkov 
(tabel), zgolj z uporabo enostavnih računskih operacij, ``ki si jih bo vsak, 
ki se mu to zdi vredno truda, z lahkoto zapomnil.'' \cite{gauss} 

Svoj algoritem je pri 23 letih objavil v članku ``Berechnung des Osterfestes 
(Izračun datuma velike noči).'' \cite{gauss}
%TODO preveri zapis naslova in narekovajev
%TODO morajo biti ti izrazi italic?
%TODO opombe na latinske izraze - 
%    Terminus Paschalis = velikonočni mejnik
%    epactae
%    Numerus aureus = zlato število
%    Littera Dominicalis = nedeljska črka
%    circulus solaris = spncev cikel
%TODO pojdi cez pravila oblikovanja in popravi vse sorte

%%%%%%%%%%%%%%%%%%%%%%%%%%%%%%%%%%%%%%%%%%%%%%%%%%%%%%%%%%%%%%%%%%%%%


\section{Kdaj sploh praznujemo veliko noč?} %TODO sprememba naslova?

Da bomo lahko razumeli Gaussov algoritem za določanje datuma velike noči, 
moramo najprej vedeti, kdaj sploh praznujemo veliko noč. 

Večino praznikov praznujemo ob obletnicah dogodkov, ki se jih spominjamo, in 
velika noč v tem smislu ni izjema. Težava pa je, da natančnega datuma vstajenja %TODO oblikuj stavek
ne poznamo. V času Jezusovega življenja so namreč na področju, kjer je živel, 
uporabljali nenatančen hebrejski koledar, ki je temeljil na opazovanju lune. 
Obsegal je 12 mesecev, pri čemer se je vsak začel ob prazni luni. Ker je po 
določenem času prišlo do zamika glede na gibanje Zemlje okoli Sonca (isti mesec 
ni vedno padel v isti letni čas), so po potrebi dodali še trinajsti mesec. To 
dodajanje ni bilo sistematično, temveč so se o njem odločali sproti, zato je za 
oddaljene datume v hebrejskem koledarju danes nemogoče določiti datum po našem 
štetju. \cite{calendar}

O datumu Jezusovega križanja se je ohranilo le to, da se je zgodilo v petek tik 
pred pasho, judovskim praznikom, ki so ga začeli praznovati 15. dne 
pomladnega meseca nisana. Ker so se meseci judovskega koledarja začenjali s 
prazno luno, se je torej zgodilo ob polni luni. Od tod pride naivna in splošno 
znana ``definicija'' tega datuma -- prva nedelja po prvi spomladanski polni 
luni. Vendar ta ``definicija'' ni natančna. Pomlad se namreč po vsem svetu ne 
začne na isti dan, pa tudi datum polne lune zaradi različnih časovnih pasov ni 
po vsem svetu isti. \cite{encyclopedia,zentralratderjuden,calendar}
%TODO enojni ali dvojni narekovaji pri definiciji

Velika noč se je do leta 325 praznovala na dan, ki ga je sproti določil papež, 
leta 325 pa so na podlagi opazovanj določili postopek za izračun datuma prve 
spomladanske polne lune in veliko noč praznovali na prvo nedeljo po tem datumu. 
Tako določeni datumi polnih lun se največkrat ujemajo z dejanskim datumom 
astronomske polne lune, v splošnem pa od njega ne odstopajo za več kot 3 dni. 
Ker je leta 325 enakonočje nastopilo 20.\ marca, so za prvi spomladanski dan 
vzeli ta datum. %TODO sklic na članek

Ker so pri določanju datuma velike noči pomembni le datumi iz napovedi, 
namesto o prvi spomladanski polni luni raje govorimo o pashalni polni luni.

Za lažje razumevanje uvedimo še pojem luninega meseca. To je čas med dvema 
zaporednima praznima lunama. %TODO definicija? - lunin mesec

\subsection*{Datum velike noči po julijanskem koledarju}

Leta 325 je bil v uporabi julijanski koledar. Tega sestavljajo običajna leta, 
dolga 365 dni, in prestopna leta, dolga 366 dni, pri čemer je vsako s 4 
deljivo leto prestopno. V povprečju tako eno leto v julijanskem koledarju 
traja 365,25 dni. Postopek določanja pashalne polne lune temelji na metonskem 
ciklu, opazki, da je v povprečju 19 let v julijanskem koledarju dolgih natanko 
toliko kot 235 luninih mesecev. Tako se datumi praznih lun in posledično tudi 
datumi pashalnih polnih lun vsakih 19 let ponavljajo. Vsakemu letu torej lahko 
glede na položaj v tem ciklu pripišemo število med 1 in 19. To število imenujemo 
\emph{zlato število} in ga izračunamo po formuli $n_a = (A \bmod 19) + 1$, pri 
čemer smo z $A$ označili leto, za katero želimo izračunati zlato število. Datum 
pashalne polne lune potem lahko preberemo iz tabele, veliko noč pa praznujemo 
na prvo nedeljo po tem datumu. %TODO definicija? - zlato število; ne uporabimo nedeljske črke - naj brišem v uvodu?

\begin{table}
    \centering
    \caption{Datumi pashalnih polnih lun po julijanskem koledarju glede na izračunano zlato število leta}
    \label{julijanski-t}
    \begin{tabular}{| c c | c c | c c |}
        \hline
        \makecell{zlato \\ število} & \makecell{pashalna \\ polna luna} & \makecell{zlato \\ število} & \makecell{pashalna \\ polna luna} & \makecell{zlato \\ število} & \makecell{pashalna \\ polna luna} \\ \hline
        1 & 5.\ april & 8 & 18.\ april & 15 & 1.\ april \\  
        2 & 25.\ marec & 9 & 7.\ april & 16 & 21.\ marec \\
        3 & 13.\ april & 10 & 27.\ marec & 17 & 9.\ april \\
        4 & 2.\ april & 11 & 15.\ april & 18 & 29.\ marec \\
        5 & 22.\ marec & 12 & 4.\ april & 19 & 17.\ april \\
        6 & 10.\ april & 13 & 24.\ marec & & \\
        7 & 30.\ marec & 14 & 12.\ april & & \\ \hline
    \end{tabular}
\end{table}

\subsection*{Datum velike noči po gregorijanskem koledarju}

Določanje datuma velike noči se zaplete leta 1582, ko so zaradi neskladja med 
dolžino Zemljinega obhoda okoli Sonca in dolžine julijanskega leta uvedli nov, 
gregorijanski koledar. Tudi tega sestavljajo običajna, 365 dni dolga leta in 
prestopna, 366 dni dolga leta, le da prestopno leto nastopi vsako s 4 deljivo 
leto, ki ni hkrati deljivo tudi s 100, pri tem pa so leta, ki so deljiva s 400, 
vseeno prestopna. Nov koledar naj bi to neskladje zmanjšal.
Poleg reforme koledarja pa so leta 1582 uvedli tudi reformo določanja datuma 
velike noči. Metonski cikel se je namreč izkazal za netočnega, poleg tega pa 
je bil z gregorijanskim koledarjem še slabše usklajen kot z julijanskim. %TODO opomba z dolžino leta in cikla

V ta namen so v postopek določanja datuma velike noči uvedli epakto. To je 
mera za starost lune na določen dan in je enaka številu dni od dneva, ko je 
luna prvič vidna (to je praviloma dan po prazni luni). V julijanskem koledarju 
epakto leta definiramo kot epakto lune na 22. marec, v gregorijanskem 
koledarju pa kot epakto lune ob začetku leta. %TODO definicija?

Epakto lahko izračunamo s pomočjo zlatega števila.
Če označimo $e' = (11 \cdot (n_a - 1)) \bmod 30$, je ob predpostavki, da za 
julijanski koledar velja metonski cikel, epakta po julijanskem koledarju enaka 
$e'$, razen v primeru, ko je $e' = 0$ in namesto 0 raje vzamemo 30, saj je bilo 
število 0 še nekaj stoletij nazaj precej nepriljubljeno. Ker to na rezultate v 
nadaljevanju ne vpliva, bomo mi v tem primeru za epakto vseeno vzeli 0.

V gregorijanskem koledarju računanje epakte prilagodimo in uvedemo nekaj 
popravkov, da bo izračunana vrednost čim bližje pravi, astronomsko določeni. 
Epakto v gregorijanskem koledarju določimo po naslednjem postopku:

\begin{enumerate}
    \item Uporabimo formulo za julijanski koledar 
        $e_j = (11 \cdot (n_a - 1)) \bmod 30$
    \item Uvedemo solarno enačbo $S = \lfloor \frac{3 \cdot C}{4} \rfloor $ 
        \footnote{Solarna enačba odraža razliko med julijanskim in gregorijanskim koledarjem in se 
        vsako leto, ki je prestopno v julijanskem, ne pa tudi v gregorijanskem 
        koledarju, poveča za 1.}
    \item Uvedemo lunarno enačbo $L = \lfloor \frac{8 \cdot C + 5}{25} \rfloor $
        \footnote{Lunarna enačba odraža napako pri uporabi metonskega cikla glede na gregorijanski 
        koledar in se vsakih 2500 let osemkrat poveča za 1.}
    \item Izračunamo $e_g' = e_j - S + L + 8$. %TODO od kje 8
    \item Podamo gregorijansko epakto $e_g' \bmod 30$, pri čemer so včasih ponovno 
        namesto 0 vzeli 30, a se bomo mi temu tudi tukaj izognili. %TODO preveri, ali še kje vzameš 30
\end{enumerate}

Datum velike noči po gregorijanskem koledarju določimo glede na epakto. Datum 
pashalne polne lune preberemo iz tabele, veliko noč pa praznujemo na nedeljo, 
ki temu datumu sledi.

\begin{table}
    \centering
    \caption{Datumi pashalnih polnih lun po gregorijanskem koledarju glede na izračunano epakto leta}
    \label{gregorijanski-t}
    \begin{tabular}{| c c | c c | c c |}
        \hline
        epakta & \makecell{pashalna \\ polna luna} & epakta & \makecell{pashalna \\ polna luna} & epakta & \makecell{pashalna \\ polna luna} \\ \hline
        1 & 12.\ april & 11 & 2.\ april & 21 & 23.\ marec \\  
        2 & 11.\ april & 12 & 1.\ april & 22 & 22.\ marec \\
        3 & 10.\ april & 13 & 31.\ marec & 23 & 21.\ marec \\
        4 & 9.\ april & 14 & 30.\ marec & 24 & 18.\ april \\
        5 & 8.\ april & 15 & 29.\ marec & 25 & 17.\ ali 18.\ april \\
        6 & 7.\ april & 16 & 28.\ marec & 26 & 17.\ april \\
        7 & 6.\ april & 17 & 27.\ marec & 27 & 16.\ april \\
        8 & 5.\ april & 18 & 26.\ marec & 28 & 15.\ april \\
        9 & 4.\ april & 19 & 25.\ marec & 29 & 14.\ april \\
        10 & 3.\ april & 20 & 24.\ marec & 0 & 13.\ april \\ \hline
    \end{tabular}
\end{table}

Kot vidimo v tabeli, sta pri epakti 25 možna dva različna datuma. 
Pravega izberemo na enega od ekvivalentnih načinov:
\begin{itemize}
    \item Če stoletje, v katerem se nahaja leto, za katerega določamo datum velike 
        noči, vsebuje leto z epakto 24, izberemo 17. april, sicer pa 18.
    \item Če je zlato število leta, za katerega določamo datum velike noči, strogo 
        večje od 11, izberemo 17. april, sicer pa 18.
\end{itemize}

Dokažimo, da sta načina res ekvivalentna. %TODO dokaz; environment?
\begin{align*}
    \exists \text{ leto z epakto } e_g = 24 &\Leftrightarrow \exists \text{ leto z } d_l = 29 \\
    &\Leftrightarrow d_l = (19 a_l + M) \bmod 30 = 29 \\
    &\Leftrightarrow 19 a_l \bmod 30 + M = 29 \\
    &\Leftrightarrow M = 29 - 19 a_l \bmod 30
\end{align*}

Dano leto $A$ ima epakto $e_g = 25$, torej $d = 28$ in ena od:
\begin{itemize}
    \item $19 a \bmod 30 = 29 \land M = 29$
    \item $M = 28 - (19 a \bmod 30)$
\end{itemize}

Če prvo: $19 a \bmod 30 = 19 (a - 11 + 11) \bmod 30 = 29 + 19 (a - 11) \bmod 30 = 29$.
Sledi: $19 (a - 11) \bmod 30 = 0$ in ob upoštevanju $0 \leq a \leq 18$ $a=11$.

Če drugo: $M$ je za obe leti enak, ker sta obe v istem stoletju, torej
\begin{align*}
    29 - 19 a_l \bmod 30 = 28 - 19 a \bmod 30 \\
    19 (a_l - a) \bmod 30 = 1 \\
    19 (a_l - a + 11 - 11) \bmod 30 = 1 \\
    19 (a_l - (a - 11)) \bmod 30 = 0 \\
    a_l - (a - 11) \bmod 30 = 0
\end{align*}

Ob upoštevanju $0 \leq a, a_l \leq 18$ dobimo $a_l = a - 11$ in $a \geq 11$ oz.\ $a > 10$.

Obrat: Če je za leto $A$ $a > 10$ in $e_g = 25$, je za to leto $d = 28$ in imata leti 
$A - 11$ in $A + 8$ epakto $e_g = 24$.
Eno izmed njiju je zagotovo v istem stoletju kot $A$.
Res, velja: $a_l = (A - 11) \bmod 19 = (A + 8) \bmod 19 = a - 11$ (upoštevali smo $a > 10$)
\begin{align*}
    e_g &= (23 - d_l) \bmod 30 \\
        &= (23 - (19 a_l + M) \bmod 30) \bmod 30 \\
        &= (23 - 19 (a - 11) - M) \bmod 30 \\
        &= (22 - 19 a - M) \bmod 30 \\
        &= (22 - (19 a + M)) \bmod 30 = (22 - 28) \bmod 30 = 24
\end{align*}

\begin{zgled}
   Določimo datum velike noči v letošnjem letu ($A = 2023$).
   Ker smo v letu po 1582, uporabimo postopek za določanje datuma velike noči 
   po gregorijanskem koledarju. 
   
   Zlato število tega leta je $$n_a = (A \bmod 19) + 1 = (2023 \bmod 19) + 1 = 10.$$
   Julijanska epakta je potem enaka 
   $$e_j = (11 \cdot (n_a - 1)) \bmod 30 = (11 \cdot (10 - 1)) \bmod 30 = 9.$$
   Ker je leto 2023 v 21.\ stoletju, je $C = 21$ in $S = \lfloor \frac{3 \cdot C}{4} \rfloor = \lfloor \frac{3 \cdot 21}{4} \rfloor = 15$ 
   ter $L = \lfloor \frac{8 \cdot C + 5}{25} \rfloor = \lfloor \frac{8 \cdot 21 + 5}{25} \rfloor = 6$.
   Potem je gregorijanska epakta enaka 
   $$e_g = (e_j - S + L + 8) \bmod 30 = (9 - 15 + 6 + 8) \bmod 30 = 8.$$
   Nato iz tabele \ref{gregorijanski-t}
   preberemo, da je bila letos pashalna polna luna 5.\ aprila, nato pa s pomočjo
   koledarja ugotovimo, da je prva nedelja po tem datumu 9.\ april.
\end{zgled}


%%%%%%%%%%%%%%%%%%%%%%%%%%%%%%%%%%%%%%%%%%%%%%%%%%%%%%%%%%%%%%%%%%%%%


\section{Gaussov algoritem}

Gauss v svojem delu podaja enostavnejši algoritem za izračun datuma velike 
noči tako po julijanskem kot tudi po gregorijanskem koledarju. Najprej vpelje 
nekaj oznak: 
\begin{align*}
    a &= A \bmod 19 \\
    b &= A \bmod 4 \\
    c &= A \bmod 7 \\
    d &= (19 \cdot a + M) \bmod 30 \\
    e &= (2 \cdot b + 4 \cdot c + 6 \cdot d + N) \bmod 7,
\end{align*}
pri čemer sta $M$ in $N$ za računanje datuma v julijanskem koledarju konstanti
$M = 15$ in $N = 6$, za računanje datuma v gregorijanskem koledarju pa ju za leta 
med $100 k$ in $100 k + 99$, pri čemer je $k \geq 16$, izračunamo na naslednji način: 
\begin{enumerate}
    \item Vpeljemo oznaki $p = \lfloor \frac{8k + 13}{25} \rfloor$ in
       $q = \lfloor \frac{k}{4} \rfloor$.
    \item Podamo $M = (15 + k - p - q) \bmod 30$ in $N = (4 + k - q) \bmod 7$
\end{enumerate}
Za leta med 1582 in 1599 vzamemo $k = 16$ in uporabimo zgornji postopek.

Gauss trdi, da pri tako definiranih spremenljivkah veliko noč praznujemo na 
$22 + d + e$. marec oz.\ v primeru, da je $22 + d + e > 31$, na $d + e - 9$. april.
V nadaljevanju bomo $n$.\ marec za $n > 31$ razumeli kot $n - 31$.\ april in ne 
bomo ločevali primerov.

Pri tem Gauss za določanje datuma velike noči v gregorijanskem koledarju navaja 
ti dve izjemi:
\begin{itemize}
    \item Če za datum velike noči dobimo 26. april, namesto tega vedno vzamemo 
        19. april.
    \item Če dobimo $d = 28$ in $e = 6$ ter število $11M+11$ pri deljenju s 
        30 da ostanek, manjši od 19, namesto 25. aprila vzamemo 18.
\end{itemize}

\begin{zgled}
    Določimo datum velike noči v letošnjem letu še z uporabo Gaussovega algoritma. 
    Ponovno je $A = 2023$. Najprej določimo $M$ in $N$. Ker smo v letu, ko je v 
    uporabi gregorijanski koledar, uporabimo formuli 
    $$M = (15 + k - \lfloor \textstyle \frac{8k + 13}{25} \rfloor - \lfloor \frac{k}{4} \rfloor) \bmod 30 = (15 + 20 - \lfloor \frac{8 \cdot 20 + 13}{25} \rfloor - \lfloor \frac{20}{4} \rfloor) \bmod 30 = 24$$ 
    in
    $$N = (4 + k - \lfloor \textstyle \frac{k}{4} \rfloor) \bmod 7 = (4 + 20 - \lfloor \frac{20}{4} \rfloor) \bmod 7 = 5.$$
    Določimo Gaussove $a$, $b$, $c$, $d$ in $e$:
    \begin{align*}
        a &= A \bmod 19 = 2023 \bmod 19 = 9 \\
        b &= A \bmod 4 = 2023 \bmod 4 = 3 \\
        c &= A \bmod 7 = 2023 \bmod 7 = 0 \\
        d &= (19 \cdot a + M) \bmod 30 = (19 \cdot 9 + 24) \bmod 30 = 15 \\
        e &= (2 \cdot b + 4 \cdot c + 6 \cdot d + N) \bmod 7 = (2 \cdot 3 + 4 \cdot 0 + 6 \cdot 15 + 5) \bmod 7 = 3
    \end{align*}
    Gaussov algoritem trdi, da veliko noč letos praznujemo $d + e - 9$.\ aprila, torej
    9.\ aprila.
\end{zgled}


%%%%%%%%%%%%%%%%%%%%%%%%%%%%%%%%%%%%%%%%%%%%%%%%%%%%%%%%%%%%%%%%%%%%%


\section{Dokaz Gaussovega algoritma}

Razmislimo, da podan algoritem velja. Začnimo pri julijanskem koledarju. 

\subsection{Dokaz Gaussovega algoritma za julijanski koledar}
Prvi korak algoritma je določanje vrednosti $a$, ki jo dobimo z deljenjem po modulu 19.
To najdemo tudi v prej opisanem postopku. Z njim namreč določimo zlato število leta.
Opazimo, da velja zveza $$n_a = a + 1.$$

Datum pashalne polne lune smo potem prebrali iz tabele \ref{julijanski-t}.

To tabelo zapišimo malo drugače, in sicer namesto datuma pashalne polne lune 
zapišimo število dni od 21.\ marca do nje.

\begin{table}
    \centering
    \caption{Število dni od 21.\ marca do pashalne polne lune po julijanskem koledarju glede na izračunano zlato število leta}
    \label{julijanski-tm}
    \begin{tabular}{| c c | c c | c c |}
        \hline
        \makecell{zlato \\ število} & \makecell{pashalna \\ polna luna} & \makecell{zlato \\ število} & \makecell{pashalna \\ polna luna} & \makecell{zlato \\ število} & \makecell{pashalna \\ polna luna} \\ \hline
        1 & 15 & 8 & 28 & 15 & 11 \\  
        2 & 4 & 9 & 17 & 16 & 0 \\
        3 & 23 & 10 & 6 & 17 & 19 \\
        4 & 12 & 11 & 25 & 18 & 8 \\
        5 & 1 & 12 & 14 & 19 & 27 \\
        6 & 20 & 13 & 3 & & \\
        7 & 9 & 14 & 22 & & \\ \hline
    \end{tabular}
\end{table}

Opazimo, da število v naslednji vrstici dobimo z alternirajočim prištevanjem 
19 in odštevanjem 11, od koder sklepamo na prištevanje števila 19 in deljenje 
po modulu 30. Ničlo najdemo pri $n_a = 16$. Datum pashalne polne lune v 
julijanskem koledarju je torej $21 + d$.\ marec, kjer je 
\begin{align*}
    d &= 19 \cdot (n_a - 16) \bmod 30 \\
    &= 19 \cdot (a - 15) \bmod 30 \\
    &= (19 \cdot a - 19 \cdot 15) \bmod 30 \\
    &= (19 \cdot a + 15) \bmod 30
\end{align*}
Opazimo, da je ta $d$ ravno Gaussov $d$.

Pokazali smo, da je datum pashalne polne lune $21 + d$.\ marec, vendar veliko noč 
praznujemo prvo nedeljo po tem datumu, tako da še nismo končali. Dobljenemu datumu 
moramo prišteti še eno od števil med (vključno) 1 in 7. Veliko noč torej praznujemo 
$21 + d + f'$. marca, kjer je $f' \in \left\{ 1, 2, 3, 4, 5, 6, 7 \right\}$, oz. 
$22 + d + f$. marca, kjer je $f \in \left\{ 0, 1, 2, 3, 4, 5, 6 \right\}$, pri 
čemer $f$ oz. $f'$ izberemo tako, da je ta dan nedelja. Opazimo, da tudi za  
parameter $e$ v Gaussovem algoritmu velja $e \in \left\{ 0, 1, 2, 3, 4, 5, 6 \right\}$.

Utemeljimo, da ta res vrne takšno število, da je $22 + d + e$. marec nedelja. 
Za začetek potrebujemo datum, za katerega vemo, da je bila nedelja. Ker 
določamo datum velike noči po julijanskem koledarju, vzamemo datum pred 
4. oktobrom 1582, npr. 19. april 1500. \cite{weekday}
Med 19. aprilom 1500 in $22 + d + e$. marcem leta $A$ je 
$$d + e - 28 + i + 365(A - 1500)$$ 
dni, kjer $i$ šteje prestopna leta med tema dvema letoma. 
\footnote{$i$ je pravzaprav število dodatnih dni, ki se zaradi prestopnih let nahajajo 
med tema dvema datumoma. 
Ker je 29.\ februar vedno pred datumom, ki ga vrne algoritem, je to enako številu prestopnih let
med letoma $A$ in 1500.}

Velja: $i = \lfloor \displaystyle \frac{(A - 1500)}{4} \rfloor = \frac{A - 1500 - (A \bmod 4)}{4}$

Število dni med 19. aprilom 1500 in $22 + d + e$. marcem leta $A$ je torej enako
$$d + e - 28 + \frac{(A - 1500 - b)}{4} + 365(A - 1500).$$
Ker želimo preveriti le, ali je $22 + d + e$.\ marec leta $A$ nedelja, je dovolj 
preveriti, ali je $$d + e - 28 + \frac{(A - 1500 - b)}{4} + 365(A - 1500) \bmod 7 = 0.$$

Velja:
\begin{align*}
    &d + e - 28 + \frac{(A - 1500 - b)}{4} + 365(A - 1500) \equiv \footnote[2]{$/+ (28 + \frac{7}{4} (A - 1500 - b))$} \\
    &\equiv d + e + 2 A - 3000 - 2b + 365(A - 1500) \equiv \footnote[3]{$/ + (550494 - 364 A)$} \\
    &\equiv d + e + 3 A - 6 - 2b \equiv \footnote[4]{$/ - (3 A - 3c)$} \\
    &\equiv d + e + 3c - 6 - 2b \equiv \footnote[5]{$/ - (7d + 7c)$} \\
    &\equiv e - 2b - 4c - 6d - 6 (\bmod 7)
\end{align*}
\footnotetext[2]{$/+ (28 + \frac{7}{4} (A - 1500 - b))$}
\footnotetext[3]{$/ + (550494 - 364 A)$}
\footnotetext[4]{$/ - (3 A - 3c)$}
\footnotetext[5]{$/ - (7d + 7c)$}

Ugotovimo, da je $22 + d + e$. marec leta $A$ nedelja le v primeru, ko je 
$(e - 2b - 4c - 6d - 6) \bmod 7 = 0$, to pa velja natanko tedaj, ko je
$e = (2b + 4c + 6d + 6) \bmod 7$, kar je ravno Gaussov $e$ za $N=6$.

S tem smo utemeljili, da Gaussov algoritem velja za julijanski koledar. 
Pokažimo, da velja tudi za gregorijanskega.


\subsection{Dokaz Gaussovega algoritma za gregorijanski koledar}

Določanje datuma velike noči tudi po gregorijanskem koledarju začnemo z določanjem zlatega 
števila leta, za katero datum velike noči določamo. Kot prej opazimo, da je $$n_a = a + 1.$$

Velja še $$e_j = (11 \cdot a) \bmod 30$$ in %TODO strah pred 0 smo že premagali, zato imamo epakto 0 (tudi v tabeli)
\begin{align*}
    e_g &= (e_j - S + L + 8) \bmod 30 \\
        &= (((11 \cdot a) \bmod 30) - S + L + 8) \bmod 30 \\
        &= (11a - S + L + 8) \bmod 30 \\
        &= \textstyle (11a - \lfloor \frac{3 (k + 1)}{4} \rfloor + \lfloor \frac{8 (k + 1) + 5}{25} \rfloor + 8) \bmod 30 \\
        &= \textstyle (11a - \frac{3 (k + 1) - (3 (k + 1) \bmod 4)}{4} + \lfloor \frac{8 (k + 1) + 5}{25} \rfloor + 8) \bmod 30 \\
        &= \textstyle (11a - \frac{3k}{4} - \frac{3}{4} + \frac{((3 k + 3) \bmod 4)}{4} + \lfloor \frac{8 (k + 1) + 5}{25} \rfloor + 8) \bmod 30 \\
        &= \textstyle (11a - \frac{3k}{4} - \frac{3}{4} + \frac{((-k + 3) \bmod 4)}{4} + \lfloor \frac{8 (k + 1) + 5}{25} \rfloor + 8) \bmod 30 \\
        &= \textstyle (11a - \frac{3k}{4} - \frac{3}{4} - \frac{(((k \bmod 4) - 3) \bmod 4)}{4} + \lfloor \frac{8 (k + 1) + 5}{25} \rfloor + 8) \bmod 30 \\
        &= \textstyle (11a - \frac{3k}{4} - \frac{3}{4} - \frac{(k \bmod 4) - 3}{4} + \lfloor \frac{8 (k + 1) + 5}{25} \rfloor + 8) \bmod 30 \\
        &= \textstyle (11a - \frac{3k}{4} - \frac{(k \bmod 4)}{4} + \lfloor \frac{8 (k + 1) + 5}{25} \rfloor + 8) \bmod 30 \\
        &= \textstyle (11a - k + \frac{k}{4} - \frac{(k \bmod 4)}{4} + \lfloor \frac{8 (k + 1) + 5}{25} \rfloor + 8) \bmod 30 \\
        &= \textstyle (11a - k + \frac{k - (k \bmod 4)}{4} + \lfloor \frac{8 (k + 1) + 5}{25} \rfloor + 8) \bmod 30 \\
        &= \textstyle (11a - k + \lfloor \frac{k}{4} \rfloor + \lfloor \frac{8 (k + 1) + 5}{25} \rfloor + 8) \bmod 30
\end{align*}

Glede na določeno epakto smo datum pashalne polne lune tudi v tem primeru prebrali 
iz tabele, ki jo ponovno zapišemo drugače, in sicer kot prej namesto datuma pashalne
polne lune pišemo število dni od 21.\ marca do nje. Dobimo:

\begin{table}
    \centering
    \caption{Število dni od 21.\ marca do pashalne polne lune po gregorijanskem koledarju glede na izračunano epakto leta}
    \label{gregorijanski-tm}
    \begin{tabular}{| c c | c c | c c |}
        \hline
        epakta & \makecell{pashalna \\ polna luna} & epakta & \makecell{pashalna \\ polna luna} & epakta & \makecell{pashalna \\ polna luna} \\ \hline
        1 & 22 & 11 & 12 & 21 & 2 \\  
        2 & 21 & 12 & 11 & 22 & 1 \\
        3 & 20 & 13 & 10 & 23 & 0 \\
        4 & 19 & 14 & 9 & 24 & 28 \\
        5 & 18 & 15 & 8 & 25 & 27 ali 28 \\
        6 & 17 & 16 & 7 & 26 & 27 \\
        7 & 16 & 17 & 6 & 27 & 26 \\
        8 & 15 & 18 & 5 & 28 & 25 \\
        9 & 14 & 19 & 4 & 29 & 24 \\
        10 & 13 & 20 & 3 & 0 & 23 \\ \hline
    \end{tabular}
\end{table}

Opazimo, da je datum pashalne polne lune (razen v primeru 
$e_g = 25$ ali $e_g = 24$) $21 + ((23 - e_g) \bmod 30)$. marec.

Če v dobljen datum vstavimo prejšnji rezultat in v njem prepoznamo Gaussove parametre,
dobimo: 
\begin{align*}
    &21 + ((23 - e_g) \bmod 30) = \\
    = &21 + ((23 - ((11a - k + \lfloor \frac{k}{4} \rfloor + \lfloor \frac{8 (k + 1) + 5}{25} \rfloor + 8) \bmod 30)) \bmod 30) \\
    = &21 + ((23 - 11a + k - \lfloor \frac{k}{4} \rfloor - \lfloor \frac{8 (k + 1) + 5}{25} \rfloor - 8) \bmod 30) \\
    = &21 + ((15 - 11a + k - p - q) \bmod 30) \\
    = &21 + ((11a + M) \bmod 30) \\
    = &21 + d
\end{align*}

Kot prej smo za datum pashalne polne lune dobili $21 + d$.\ marec. Podobno kot prej 
utemeljimo tudi, da je izbira $e$ prava. Tokrat moramo za izhodišče izbrati neki 
datum po 15. oktobru 1582. Kot Gauss bomo tudi mi vzeli nedeljo, 21. marec 1700. \cite{gauss,weekday}

Število dni med 21. marcem 1700 in $22 + d + e$. marcem leta $A$ je enako 
$$d + e + 1 + i + 365(A - 1700),$$ 
kjer $i$ ponovno šteje prestopna leta med tema dvema datumoma, 
\footnote{$i$ je pravzaprav število dodatnih dni zaradi prestopnih let.} 
vendar v tem primeru
(ob upoštevanju zaporedja prestopnih in neprestopnih let v gregorijanskem koledarju)
velja $$i = \textstyle \frac{A - 1700 - (A \bmod 4)}{4} - C + 18 + \frac{C - 17 - ((C - 1) \bmod 4)}{4}.$$

Pri tem smo s $C$ označili stoletje, v katerem se nahaja leto $A$.

%TODO številčenje opomb

Ker želimo preveriti le, ali je $22 + d + e$.\ marec leta $A$ nedelja, je dovolj 
preveriti, ali je $$d + e + 1 + i + 365(A - 1700) \bmod 7 = 0.$$

Velja: 
\begin{align*}
    &d + e + 1 + \textstyle \frac{A - 1700 - (A \bmod 4)}{4} - C + 18 + \frac{C - 17 - ((C - 1) \bmod 4)}{4} + 365 (A - 1700) \equiv \\
    &\equiv d + e + 20 + \textstyle \frac{A - 1700 - b}{4} - k + \frac{k - 16 - (k \bmod 4)}{4} + 365(A - 1700) \equiv \\
    &\equiv d + e + 20 + \textstyle \frac{A - b}{4} - 425 - k + \lfloor \frac{k}{4} \rfloor - 4 + 365A - 2065 \equiv \footnote[7]{$/ \frac{7}{4} (A - b)$} \\
    &\equiv d + e + 4 + 3A - 2b - k + q \equiv \footnote[8]{$/ -(3A - 3c)$} \\
    &\equiv e - (2 b + 4 c + 6 d + N) (\bmod 7)
\end{align*}
\footnotetext[7]{$/ \frac{7}{4} (A - b)$}
\footnotetext[8]{$/ -(3A - 3c)$}

Ugotovimo, da je $22 + d + e$. marec leta $A$ nedelja le v primeru, ko je 
$e = (2b + 4c + 6d + 6) \bmod 7$, kar je ravno Gaussov $e$.

Ustavimo se še pri izjemah, ki ju navaja Gaussov algoritem za izračun datuma 
velike noči po gregorijanskem koledarju. Spomnimo se, da smo pri izpeljavi 
formule v gregorijanskem koledarju predpostavili, da $e_g$ ni enak 24 ali 25. 
Ta dve izjemi pokrivata ravno ta dva primera.

Prva izjema pravi: ``Če za datum velike noči dobimo 26.\ april, namesto tega vedno 
vzamemo 19.\ april.''

Gaussov algoritem vrne 26. april natanko tedaj, ko je $d + e - 9 = 26$. Ker je 
$d$ ostanek pri deljenju s 30, $e$ pa ostanek pri deljenju s 7, je edina možnost 
za to $d = (19 a + M) \bmod 30 = 29$ in $e = 6$. 
Potem velja:
\begin{align*}
    e_g &= (11a - k + p + q + 8) \bmod 30 = \\
        &= (-19 a + 23 - (15 + k - p - q)) \bmod 30 = \\
        &= (23 - (19 a + M)) \bmod 30 = \\
        &= (23 - ((19 a + M) \bmod 30)) \bmod 30 = \\
        &= (23 - d) \bmod 30 \\
        &= (23 - 29) \bmod 30 \\
        &= 24 %TODO vsakič nova vrstica?
\end{align*}
V tem primeru je torej epakta 24.

Velja tudi obratno, če je gregorijanska epakta leta $e_g = (23 - d) \bmod 30 = 24$, 
je za to leto $d = 29$.

Za $d = 29$ oz.\ $e_g = 24$ je datum pashalne polne lune po tabeli 18.\ april, 
Gaussov algoritem pa za datum pashalne polne lune določi $d - 10$.\ april, torej 
19. april. Ker veliko noč praznujemo na prvo nedeljo po tem datumu, se bosta 
izračunana datuma velike noči razlikovala le v primeru, ko je 19.\ april nedelja 
oz.\ Gaussov $e = 6$.
Takrat bo namreč Gaussov algoritem za datum velike noči vrnil 26.\ april, pravi 
datum velike noči glede na tabele pa je 19.\ april. Prva izjema torej popravlja 
ravno to napako.

Druga izjema pravi: ``Če dobimo $d = 28$ in $e = 6$ ter število $11M + 11$ pri deljenju
s 30 da ostanek, manjši od 19, namesto 25. aprila vzamemo 18.''

Razmislimo, kdaj dobimo $d=28$. Velja:
\begin{align*}
    d &= (19 a + M) \bmod 30 = \\
    &= ((19 a \bmod 30) + (M \bmod 30)) \bmod 30, 
\end{align*}
torej imamo dve možnosti:
\begin{itemize} %TODO enumerate?
    \item $19 a \bmod 30 = M \bmod 30 = 29$
    \item $(19 a \bmod 30) + (M \bmod 30) = 28$
\end{itemize}

V prvem primeru je $(11M + 11) \bmod 30 = (11 \cdot 29 + 11) \bmod 30 = 0$ in 
$19 a \bmod 30 = 29$, torej velja: 
\begin{align*}
    19 a \bmod 30 &= 19 (a - 11 + 11) \bmod 30 = \\
        &= 29 + (19 (a - 11) \bmod 30) = 29, 
\end{align*}
$a = 11$ in $n_a = 12$.

V drugem primeru je $M \bmod 30 = 28 - (19 a \bmod 30)$, torej velja:
\begin{align*}
    (11 M + 11) \bmod 30 &= (11 (28 - (19 a \bmod 30)) + 11) \bmod 30 = \\
        &= (19 - (11 \cdot 19 a \bmod 30)) \bmod 30 = \\
        &= (19 - 11 \cdot 19 a) \bmod 30 = \\
        &= (19 - 11 \cdot 19 \cdot (a - 11 + 11)) \bmod 30 = \\
        &= (-11 \cdot 19 \cdot (a - 11)) \bmod 30 = \\
        &= (a - 11) \bmod 30
\end{align*}

Opazimo, da je $(11M + 11) \bmod 30 = (a - 11) \bmod 30 < 19$ natanko tedaj, ko je 
$a > 10$ oz.\ $n_a = a + 1 > 11$.

V obeh primerih je torej $(11M + 11) \bmod 30 < 19$ natanko tedaj, ko je $n_a > 11$.

Kot prej iz enačbe $e_g = (23 - d) \bmod 30$ sklepamo, da je $d = 28$ natanko tedaj, 
ko je gregorijanska epakta leta enaka 25.

Spomnimo se, da smo pri epakti 25 v tabeli \ref{gregorijanski-t}
imeli dva možna datuma pashalne polne lune, 
pri čemer smo pravega izbrali glede na zlato število. Če je veljalo $n_a > 11$, smo 
za datum pashalne polne lune vzeli 17.\ april, sicer pa 18.\ april.

Gaussov algoritem ne glede na velikost zlatega števila v tem primeru za datum 
pashalne polne lune vedno vrne $d - 10$.\ april oz.\ 18.\ april. Razlika v 
izračunanem datumu pashalne polne lune se torej pojavi le v primeru, ko je $n_a > 11$
oz.\ ekvivalentno $(11M + 11) \bmod 30 < 19$. Podobno kot prej razmislimo, da to 
na datum velike noči vpliva le v primeru, ko je 18.\ april nedelja oz.\ ko je 
Gaussov parameter $e = 6$. %TODO parameter?

Druga izjema Gaussovega algoritma popravlja to napako.


\begin{zgled}
    Za konec določimo še Gaussov rojstni dan. 
\end{zgled} %TODO primer

\nocite{*}

\bibliographystyle{siam} %TODO literatura
\bibliography{gaussov-algoritem-za-racunanje-datuma-velike-noci_clanek}

\end{document}
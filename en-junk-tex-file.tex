\begin{document}
%%%%%%%%%%%%%%%%%%%%%%%%%%%%%%%%%%%%%%%%%%%%%%%%%%%%%%%%%%%%%%%%%%%%%
%%%%%%%%%%%%%%%%%%%%%%%%%%%%%%%%%%%%%%%%%%%%%%%%%%%%%%%%%%%%%%%%%%%%%
%%%%%%%%%%%%%%%%%%%%%%%%%%%%%%%%%%%%%%%%%%%%%%%%%%%%%%%%%%%%%%%%%%%%%



Najprej se moramo dogovoriti, kaj sploh pomeni \emph{prva spomladanska} 
polna luna. 
%
%TODO Pomlad namreč po vsem svetu ne nastopi istočasno. Ko se na 
% severni polobli začne pomlad, se na južni začne jesen. Tu je seveda mišljen
% začetek pomladi na severni polobli, ki nastopi konec marca. Velika noč se 
% tudi na južni polobli praznuje na enak dan kot na severni (in ne v septembru 
% oz.\ oktobru).
%
Astronomsko se pomlad začne ob spomladanskem enakonočju, torej ob trenutku, 
ko ekvatorialna ravnina Zemlje seka središče Sonca. To se vsako leto zgodi 
med 18.\ in 22.\ marcem. V definiciji datuma velike noči, ki jo uporablja 
Katoliška cerkev, tega ne upoštevamo in datum določamo glede na prvo 
polno luno po 20.\ marcu, ki ni nujno tudi prva polna luna po astronomskem
začetku pomladi. Za določanje datuma velike noči relevantna polna luna tako 
nastopi najhitreje 21.\ marca.

Vendar tu težav še ni konec. Dogovoriti se moramo še, kaj je to \emph{polna
luna}. Astronomsko je to trenutek, ko se središče Zemlje nahaja na zveznici 
med središčem Sonca in središčem Lune. Ker imamo na Zemlji različne časovne 
pasove, lahko ta trenutek v dveh različnih časovnih pasovih pade na različna 
datuma. Da bi se izognili zmedi, se je Katoliška cerkev glede na napovedi 
astronomov junija 325 dogovorila za postopek izračuna tega datuma in 
nekatere rezultate zapisala v tabele. K temu postopku se bomo še vrnili pri
utemeljevanju Gaussovega algoritma. Tako določeni datumi se največkrat 
ujemajo z dejanskim datumom astronomske polne lune, v splošnem pa od njega ne 
odstopajo za več kot 3 dni. Velika noč se tako po vsem svetu praznuje na 
isti dan ne glede na časovni pas, ki mu kraj pripada.

Polno luno, ki jo upoštevamo pri določanju datuma velike noči (torej prvo 
polno luno po 20.\ marcu glede na ta postopek) imenujemo \emph{pashalna 
polna luna}. Veliko noč tako praznujemo na prvo nedeljo po pashalni polni luni.
Glavni problem pri določanju datuma velike noči je torej določiti datum 
pashalne polne lune.

%TODO preveri začetnice sonce luna zemlja
%TODO glede na gregorijanski koledar
%TODO dodaj tabelo


%%%%%%%%%%%%%%%%%%%%%%%%%%%%%%%%%%%%%%%%%%%%%%%%%%%%%%%%%%%%%%%%%%%%%


\section{Postopek}

Določanje datuma velike noči po julijanskem koledarju je dokaj enostavna 
naloga. Julijanski koledar sestavljajo običajna leta, dolga 365 dni, in 
prestopna leta, dolga 366 dni, pri čemer je vsako s 4 deljivo leto 
prestopno. V povprečju tako eno leto v julijanskem koledarju traja 365,25
dni, kar je približno enako dolžini enega tropskega leta (času med 
zaporednima zimskima solsticijema). %TODO definicija tropskega leta

Že v starem veku so vedeli, da 19 tropskih let traja skoraj natanko toliko 
kot 235 luninih mesecev (čas med zaporednima praznima lunama). Ker je 
dolžina leta v julijanskem koledarju približno enaka dolžini tropskega leta, 
od tod sledi, da se zaporedje datumov v julijanskem koledarju, na katere bo 
luna prazna, vsakih 19 let periodično ponavlja. Ker nas pri določanju datuma
velike noči zanima le prva polna luna po 20.\ marcu, polna luna pa nastopi 
13 dni po prazni, je vsako leto relevanten le datum prve prazne lune po 7.\ 
marcu. Dobimo zaporedje %TODO iz vira, lahko tabela

Ker se to vsakih 19 let periodično ponovi, lahko letu glede na položaj v tem 
ciklu pripišemo število med 1 in 19. To število imenujemo \emph{zlato število} 
in ga izračunamo po formuli $a = (l \bmod 19) + 1$, kjer smo leto označili $l$ 
in zlato število $a$.

Zaporedje datumov praznih lun v letih, katerih zlata števila se ujemajo, je
(približno) enako. Zgornje ocene tega datuma uporabimo pri izračunu datuma 
pashalne polne lune v julijanskem koledarju. Če poznamo zlato število danega 
leta, lahko datum pashalne polne lune najdemo v tabeli. 
Velikonočna nedelja je potem prva nedelja za tem datumom.

Ker se dolžina leta po julijanskem koledarju ni skladala z dolžino tropskega
leta, so v koledar uvedli nekatere popravke. Nov, gregorijanski koledar tako
kot julijanskega sestavljajo običajna, 365 dni dolga leta in prestopna, 366
dni dolga leta, le da prestopno leto nastopi vsako s 4 deljivo leto, ki ni 
hkrati deljivo tudi s 100, pri tem pa so leta, ki so deljivo s 400, vseeno 
prestopna. 

Izračun datuma velike noči pa gregorijanskem koledarju je bolj zapleten. 
Poleg reforme koledarja so namreč uvedli tudi reformo postopka določanja 
datuma velike noči. Metonski cikel se je namreč izkazal za netočnega, poleg 
tega pa je ta z gregorijanskim koledarjem še slabše usklajen kot z julijanskim.

%TODO opomba z dolžino luninega meseca
% Če to oceno
% prenesemo v julijanski koledar, dobimo dolžino luninega meseca 
% $\frac{19 \cdot 365,25}{235} = 29,53085$ dni.
% Prava dolžina luninega meseca je približno 29,53059 dni.

V ta namen so v postopek določanja datuma velike noči uvedli epakto. To je 
mera za starost lune na določen dan in je enaka številu dni od dneva, ko je 
luna prvič vidna (to je praviloma dan po prazni luni).
V julijanskem koledarju epakto leta definiramo kot epakto lune na 22.\ marec,
v gregorijanskem koledarju pa kot epakto lune ob začetku leta. 

Epakto lahko izračunamo s pomočjo zlatega števila.
Ob predpostavki, da za julijanski koledar velja metonski cikel, velja 
naslednja relacija: $e' = (11 \cdot (a - 1)) \bmod 30$
Epakta je potem enaka $e'$ razen v primer, ko je $e' = 0$ in namesto 0 raje 
vzamemo 30 (včasih ljudje res niso marali ničle).
Ker imamo le 19 možnih zlatih števil, je tudi različnih vrednosti epakte 19 
-- 1, 3, 4, 6, 7, 9, 11, 12, 14, 15, 17, 18, 20, 22, 23, 25, 26, 28 in 30.

%TODO pomen oznak
%TODO definiraj metonski cikel

V gregorijanskem koledarju računanje epakte prilagodimo in uvedemo nekaj 
popravkov, da bo izračunana vrednost čim bližje pravi, astronomsko določeni. 
Epakto v gregorijanskem koledarju določimo po spodnjem postopku, pri čemer so
vsa deljenja celoštevilska (vzamemo spodnji celi del količnika).
\begin{enumerate}
   \item Uporabimo formulo za julijanski koledar $e_j = (11 \cdot (a - 1)) \bmod 30$
   \item Uvedemo solarno enačbo $S = \frac{3 \cdot cen}{4}$, kjer smo s $cen$
      označili stoletje, ki mu pripada leto, katerega datum velike noči 
      računamo. Ta odraža razliko med gregorijanskim in julijanskim koledarjem
      in se vsako leto, ki ni prestopno, poveča za 1.
   \item Uvedemo lunarno enačbo $L = \frac{8 \cdot cen + 5}{25}$, ki odraža
      napako pri uporabi metonskega cikla v julijanskem koledarju in se vsakih
      2500 let osemkrat poveča za 1
   \item Izračunamo $e_g' = e_j - S + L + 8$. Pri tem je 8 konstanta, ki 
      jo potrebujemo zaradi definicije gregorijanske epakte na novo leto, da dejansko pade na novo leto
   \item Gregorijanska epakta je potem enaka $e_g' \bmod 30$, pri čemer ponovno
      namesto 0 vzamemo 30.
\end{enumerate}
Kljub temu da je možnih julijanskih epakt le 19, pa gregorijanska epakta lahko
zavzame poljubno vrednost med 1 in 30.

Datum velike noči po gregorijanskem koledarju potem določimo glede na epakto.
Datum pashalne polne lune preberemo iz tabele, veliko noč pa praznujemo na 
nedeljo, ki temu datumu sledi.


%%%%%%%%%%%%%%%%%%%%%%%%%%%%%%%%%%%%%%%%%%%%%%%%%%%%%%%%%%%%%%%%%%%%%


\section{Kdaj sploh praznujemo veliko noč?}

Da bomo lahko razumeli Gaussov algoritem za določanje datuma velike noči, 
moramo najprej vedeti, kdaj sploh praznujemo veliko noč. Naivna in splošno 
znana ``definicija'' tega datuma je prva nedelja po prvi spomladanski polni 
luni, a ta ni natančna.
%TODO enojni ali dvojni narekovaji pri definiciji

Če želimo podati natančnejšo definicijo, se moramo dogovoriti, kako merimo 
čas oz.\ izbrati ustrezen koledar.

Koledarji navadno temeljijo na astronomskih opazovanjih.
V času Jezusovega živeljenja so na področju, kjer je živel, uporabljali 
nenatančen hebrejski koledar, ki je temeljil na opazovanju lune. Obsegal je
12 mesecev, pri čemer se je vsak začel ob prazni luni. Ker je po določenem
času prišlo do zamika glede na gibanje Zemlje okoli Sonca (isti mesec ni vedno 
padel v isti letni čas), so po potrebi dodali še trinajsti mesec. To 
dodajanje ni bilo sistematično, temveč so se o njem odločali sproti. 
Zato je za oddaljene datume v hebrejskem koledarju danes nemogoče določiti 
datum po našem štetju.

O datumu Jezusovega križanja se je ohranilo le to, da se je zgodilo tik 
pred pasho, judovskim praznikom, ki so ga začeli praznovati 15. dne 
pomladnega meseca nisana.
%TODO moram razlagati astronomske pojme? prazna luna, polna luna


%%%%%%%%%%%%%%%%%%%%%%%%%%%%%%%%%%%%%%%%%%%%%%%%%%%%%%%%%%%%%%%%%%%%%


\section{Kdaj sploh praznujemo veliko noč?}

Da bomo lahko razumeli Gaussov algoritem za določanje datuma velike noči, 
moramo najprej vedeti, kdaj sploh praznujemo veliko noč. Naivna in splošno 
znana ``definicija'' tega datuma je prva nedelja po prvi spomladanski polni 
luni, a ta ni natančna.
%TODO enojni ali dvojni narekovaji pri definiciji

Sprva je datum velike noči določal papež, leta 325 pa so se odločili, da 
poiščejo bolj sistematičen način določanja datuma velike noči. Ta je 
temeljil na astronomskih opazovanjih
%TODO moram razlagati astronomske pojme? prazna luna, polna luna


%%%%%%%%%%%%%%%%%%%%%%%%%%%%%%%%%%%%%%%%%%%%%%%%%%%%%%%%%%%%%%%%%%%%%

\section{Gaussov algoritem za določanje datuma velike noči}

Gauss v svojem delu navaja naslednji algoritem za izračun datumov velike noči v
letih med 1700 in 1899:
\begin{enumerate}
   \item Najprej leto, za katero želimo izračunati datum velike noči, delimo z
      19, 4 in 7 ter ostanke pri deljenju zaporedoma označimo z $a$, $b$ in $c$.
      Na kvociente se ne oziramo. Enako velja tudi za vsa ostala deljenja tega
      algoritma.
   \item Število $19 \cdot a + 23$ delimo s 30 in ostanek pri deljenju označimo z $d$.
   \item Za leta med 1700 in 1799 s 7 delimo število $2 \cdot b + 4 \cdot c + 6 \cdot d + 3$,
      za leta med 1800 in 1899 pa število $2 \cdot b + 4 \cdot c + 6 \cdot d + 4$. 
      V obeh primerih ostanek pri deljenju označimo z $e$.
\end{enumerate}
%TODO označimo a ali označimo z a?
Veliko noč potem praznujemo $22 + d + e$.\ marca oz.\ , kadar je število $d + e$ 
večje od 9, $d + e - 9$.\ aprila.
%TODO 22 + d + e.? slogovno kako pika?
%TODO oz., ??S

\begin{zgled}
   Gaussov rojstni dan
\end{zgled}


%%%%%%%%%%%%%%%%%%%%%%%%%%%%%%%%%%%%%%%%%%%%%%%%%%%%%%%%%%%%%%%%%%%%%


\section{Metonski cikel}

Že v starem veku so vedeli, da 19 tropskih let (čas med zaporednima
zimskima solsticijema) traja skoraj natanko toliko kot 235 luninih mesecev 
(čas med zaporednima praznima lunama). 

Julijanski koledar sestavljajo običajna leta, dolga 365 dni, in prestopna
leta, dolga 366 dni. V povprečju tako eno leto v julijanskem koledarju traja
365,25 dni, kar je približno toliko kot dolžina enega tropskega leta. Od tod 
dobimo dolžino luninega meseca $\frac{19 \cdot 365,25}{235} = 29,53085$ dni.
Prava dolžina luninega meseca je približno 29,53059 dni.

Če začnemo s poznano prazno luno, lahko v alternirajočih razmikih po 30 in 29 
dni razporedimo anslednje prazne lune za naslednjih 19 let. Ta vzorec poda
povprečno dolžino luninega meseca 29,5 dni, kar je malo prekratko, zato 
dodamo ``prestopne meesece'' -- 6 mesecev po 30 dni in enega po 29. Te 
vstavimo v izbrane intervale, da zaporedje metonskih praznih lun ohranimo čim
bližje datumov dejanskih polnih lun. Metonski cikel trdi, da se ta vzorec 
ponovi natanko vsakih 19 let.

Na podlagi metonskih praznih lun določimo zaporedje prvih polnih lun po 20.\ 
marcu (le te so pomembne za določitev datuma velike noči; važne so tiste prazne, 
ki jim sledi polna luna po 20 marcu). V devetnajstletnem 
ciklu te padejo na:
%TODO kdaj -- tabela
Gaussov a torej določa le, kam v ta cikel pade želeno leto. (leto mod 19 = a),
torej katera od zgoraj naštetih polnih lun je relevantna.

Na podlagi te zveze
 \emph{Metonski cikel}.

%TODO - razlaga za lunine mesece in sončeva leta glede na sonce/zvezde oz. glede na zemljo/sonce


%%%%%%%%%%%%%%%%%%%%%%%%%%%%%%%%%%%%%%%%%%%%%%%%%%%%%%%%%%%%%%%%%%%%%


\section{Koledarji}

Julijanski, gregorijanski, judovski


Da pa bomo lahko
odgovorili na to vprašanje, se moramo dogovoriti, kako merimo čas oz.\ 
izbrati ustrezen koledar.

Koledarji navadno temeljijo na astronomskih opazovanjih. Pri tem so pomembni
naslednji pojmi:
\begin{definicija}
   \emph{Enakonočje} ali \emph{ekvinokcij} je trenutek, ko ekvatorialna 
   ravnina Zemlje seka središče Sonca.
\end{definicija}
V enem obhodu Zemlje okoli Sonca se to zgodi dvakrat. Tako ločimo 
\begin{definicija}
   soncevo leto
\end{definicija}
\begin{definicija}
   lunin mesec
\end{definicija}
\begin{definicija}
   polna luna
\end{definicija}

V času Jezusovega življenja so na področju, kjer je živel, uporabljali 
nenatančen hebrejski koledar. Ta se 



\section*{viri}
Datum velike noči:

https://web.archive.org/web/20081217101213/http://www.assa.org.au/edm.html

https://web.archive.org/web/20081029062823/http://users.sa.chariot.net.au/~gmarts/easter.htm

https://www.newadvent.org/cathen/05224d.htm search Sanhedrin

https://web.archive.org/web/20081103111329/http://www.smart.net/~mmontes/ec-cal.html

https://web.archive.org/web/20081002031029/http://www.tondering.dk/claus/calendar.html



Metonski cikel:

https://de.wikipedia.org/wiki/Meton-Zyklus

\end{document}